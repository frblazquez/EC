%   Copyright © 2019
%
% Francisco Javier Blázquez Martínez ~ frblazqu@ucm.es
% 
% Double degree in Mathematics-Computer engineering
% Complutense University, Madrid

\documentclass[12pt]{article}
 
\usepackage[margin=1in]{geometry} 
\usepackage{amsmath,amsthm,amssymb}
\usepackage[margin=1in]{geometry} 
\usepackage{amsmath,amsthm,amssymb}
\usepackage[spanish]{babel}     %Castellanización
\usepackage[T1]{fontenc}        %Escribe lo del teclado
\usepackage[utf8]{inputenc}     %Reconoce algunos símbolos
\usepackage{lmodern}            %optimiza algunas fuentes
\usepackage{graphicx}
\graphicspath{ {images/} }
\usepackage{hyperref} % Uso de links

\setlength{\parindent}{0cm}
\setlength{\parskip}{0cm}

\newenvironment{solution}{\begin{proof}[Solution]}{\end{proof}}

%--------------------------------------------------------------------------
%   PARA INCLUIR CÓDIGO C++ EN EL DOCUMENTO
%--------------------------------------------------------------------------
\usepackage{color}
\definecolor{gray97}{gray}{.97}
\definecolor{gray75}{gray}{.75}
\definecolor{gray45}{gray}{.45}

\usepackage{listings}
\lstset{ frame=Ltb,
framerule=0pt,
aboveskip=0.5cm,
framextopmargin=3pt,
framexbottommargin=3pt,
framexleftmargin=0.4cm,
framesep=0pt,
rulesep=.4pt,
backgroundcolor=\color{gray97},
rulesepcolor=\color{black},
%
stringstyle=\ttfamily,
showstringspaces = false,
basicstyle=\small\ttfamily,
commentstyle=\color{gray45},
keywordstyle=\bfseries,
%
numbers=left,
numbersep=15pt,
numberstyle=\tiny,
numberfirstline = false,
breaklines=true,
}

% minimizar fragmentado de listados
\lstnewenvironment{listing}[1][]
{\lstset{#1}\pagebreak[0]}{\pagebreak[0]}

\lstdefinestyle{consola}
{basicstyle=\scriptsize\bf\ttfamily,
backgroundcolor=\color{gray75},
}

\lstdefinestyle{C}
{language=C,
}

%--------------------------------------------------------------------------
 
\begin{document}
 
% --------------------------------------------------------------
%                         Start here
% --------------------------------------------------------------
 
\title{\textbf{Práctica 1} \\ 
       \large Estructuras de Computadores}
\author{Francisco Javier Blázquez Martínez}
\maketitle

\setlength{\parskip}{\baselineskip}

\section{Conceptos preliminares}

Damos en esta sección una introducción al proceso que lleva un programa en un
lenguaje de alto nivel como \textit{C} a instrucciones ejecutables en un
determinado procesador, como será el caso del ARM7TDMI.

Partiendo de un fichero .c de código fuente podemos solicitarle a un compilador 
(analizaremos en concreto el compilador GCC) que genere un ejecutable, esto es,
una secuencia de instrucciones del repertorio del procesador (y en binario, en
código máquina) que realicen lo esperado del propio código del fichero .c. Para esto el compilador realiza una serie de etapas:

\begin{enumerate}
    \item \textbf{Preprocesado:} 
    Las directivas al procesador comienzan con '\#' y son del tipo de los
    \#include \#define ... más en https://es.wikipedia.org/wiki/Preprocesador\_de\_C.
    
    \item \textbf{Compilación:}
    Transforma el código C en lenguaje ensamblador propio de la máquina donde va a 
    ser ejecutado. En el caso de GCC este se puede consultar.
    
    \item \textbf{Ensamblado:}
    En esta etapa se da el paso de código en ensamblador a código objeto (código
    máquina). Secuencia de instrucciones ejecutables por el procesador.
          
    \item \textbf{Enlazado:}
    Nos permite unir distintas funciones y funcionalidades que pueden estar presentes
    en otras zonas de memoria por haberse incluido de librerías propias del lenguaje
    o de ficheros realizados por el usuario y compilados independientemente.
    
\end{enumerate}

Para más información sobre el compilador GCC, sus comandos y sus distintas 
etapas en la compilación ver [ referencia ].
https://iie.fing.edu.uy/~vagonbar/gcc-make/gcc.htm\#Sintaxis.


\section{Script de enlazado y mapa de memoria}

El mapa de memoria viene a ser la organización y situación final en memoria del 
programa que nosotros desarrollemos. Esto es, dónde situaremos nuestro puntero
de pila inicialmente, dónde almacenaremos nuestras variables globales e incluso,
donde se almacenará nuestro propio programa una vez que sea puramente código
máquina. El script de enlazado se incluye abajo (será en nuestros proyectos el
fichero \textit{ld\_script.ld}).

\begin{listing}
SECTIONS {
    . = 0x0C000000;
    .data : {
      *(.data)
      *(.rodata)
}
.bss : {
*(.bss)
*(COMMON) }
.text : {
*(.text) }
    PROVIDE(end = .);
    PROVIDE(_stack = 0x0C7FF000 );
}
\end{listing}

En este indicamos las direcciones en memoria de nuestro fichero ensamblador de 
instrucciones de inicialización para la ejecución del programa. Este se puede ver
abajo. Esto la verdad es que no lo entiendo bien bien, vale por una buena tutoría.

\begin{listing}
.extern main
.global start

.equ USRSTACK, 0xc7ff000

.text
start:
	ldr sp, =USRSTACK
    mov fp, #0
    bl main

End:
    B End
.end
\end{listing}

\section{Variables enteras}

\section{Variables tipo carácter}

\section{Variables en punto flotante}

\section{Aritmética entera}

\section{Aritmética en coma flotante}

\section{Conversión de tipos}

\section{Tipos definidos}




% --------------------------------------------------------------
%     You don't have to mess with anything below this line.
% --------------------------------------------------------------
 
\end{document}