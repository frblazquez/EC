%   Copyright © 2019
%
% Francisco Javier Blázquez Martínez ~ frblazqu@ucm.es
% 
% Double degree in Mathematics-Computer engineering
% Complutense University, Madrid

\documentclass[12pt]{article}
 
\usepackage[margin=1in]{geometry} 
\usepackage{amsmath,amsthm,amssymb}
\usepackage[margin=1in]{geometry} 
\usepackage{amsmath,amsthm,amssymb}
\usepackage[spanish]{babel}     %Castellanización
\usepackage[T1]{fontenc}        %Escribe lo del teclado
\usepackage[utf8]{inputenc}     %Reconoce algunos símbolos
\usepackage{lmodern}            %optimiza algunas fuentes
\usepackage{graphicx}
\graphicspath{ {images/} }
\usepackage{hyperref} % Uso de links
 

\newenvironment{solution}{\begin{proof}[Solution]}{\end{proof}}

%--------------------------------------------------------------------------
%   PARA INCLUIR CÓDIGO C++ EN EL DOCUMENTO
%--------------------------------------------------------------------------
\usepackage{color}
\definecolor{gray97}{gray}{.97}
\definecolor{gray75}{gray}{.75}
\definecolor{gray45}{gray}{.45}

\usepackage{listings}
\lstset{ frame=Ltb,
framerule=0pt,
aboveskip=0.5cm,
framextopmargin=3pt,
framexbottommargin=3pt,
framexleftmargin=0.4cm,
framesep=0pt,
rulesep=.4pt,
backgroundcolor=\color{gray97},
rulesepcolor=\color{black},
%
stringstyle=\ttfamily,
showstringspaces = false,
basicstyle=\small\ttfamily,
commentstyle=\color{gray45},
keywordstyle=\bfseries,
%
numbers=left,
numbersep=15pt,
numberstyle=\tiny,
numberfirstline = false,
breaklines=true,
}

% minimizar fragmentado de listados
\lstnewenvironment{listing}[1][]
{\lstset{#1}\pagebreak[0]}{\pagebreak[0]}

\lstdefinestyle{consola}
{basicstyle=\scriptsize\bf\ttfamily,
backgroundcolor=\color{gray75},
}

\lstdefinestyle{C}
{language=C,
}

%--------------------------------------------------------------------------
 
\begin{document}
 
% --------------------------------------------------------------
%                         Start here
% --------------------------------------------------------------
 
\title{\textbf{Práctica 1} \\ 
       \large Estructuras de Computadores}
\author{Francisco Javier Blázquez Martínez}
\maketitle

\section{Conceptos preliminares}

\section{Script de enlazado}

\section{Variables enteras}

\section{Variables tipo carácter}

\section{Variables en punto flotante}

\section{Aritmética entera}

\section{Aritmética en coma flotante}

\section{Conversión de tipos}

\section{Tipos definidos}




% --------------------------------------------------------------
%     You don't have to mess with anything below this line.
% --------------------------------------------------------------
 
\end{document}